\documentclass[a4paper]{article}

\usepackage{amsmath}
\usepackage{graphicx}
\usepackage{hyperref}
\usepackage{tabularx}

\title{Forensics analysis of USB device traces\\
\large Digital Forensics}

\author{
\begin{tabular}{>{\raggedleft}m{5cm}m{5cm}}
I. Duits & (1876171) \\
E. Geretto & (1869426) \\
G. Iadarola & (1879480) \\
\end{tabular}
}

\begin{document}
\maketitle

\section{Introduction}
In digital forensics it is necessary to inspect certain computer systems to see what information they hold and how they were used. Knowledge about when a USB device was connected can be helpful in the investigation. USB devices can be used to transfer valuable data in cases of data theft or possession of illegal material. Copying data leaves traces, as A. Ramani and S. Dewangan demonstrated in their paper \cite{Abhijeet14} and T. Roy and A. Jain in theirs.~\cite{Tanushree12} 

In Section~\ref{sec:lit}, we show how to access this information in Windows and Linux systems. There are tools which can help for Windows. However, for Linux there are no tools like that. Linux is getting slowly more private users\cite{osShare} and a lot of Internet servers run on a Linux distribution\cite{InternetServer}. 

can get information from the log files on a Windows system, especially for an image of a disk, which is most used in digital forensics. 

However, in this paper we will not focus on Windows machines or how data is copied. We will focus on how we can detect the use of a usb device on Linux. There is not a easy way to do this for a images created for a Linux machine.




\section{Log files in the main operating system}
\label{sec:lit}
Log files are one of the most important sources of information when trying to
reconstruct the events that lead to a certain situation in a computer system.
Indeed, they are constructed recording a series of timestamped messages,
generated by the various components of the system, that, when considered as a
whole, give a complete overview of the operations in execution at a given
moment.

Given the information they contain, log files are also one of the most important
sources of information during a forensics investigation since they allow to
trace possibly malicious activity through time. As a consequence, it is
essential to be able to retrieve and analize them in the fastest way
possible.~\cite{finlayson1987log}

One of the main problems when analyzing log files is that they tend to be huge
in terms of the number of messages collected in them. For this reason, forensics
investigators usually rely on tools that automatize the process of extraction
and parsing in order to extract immediately the relevant information.

Given that different operating systems generate log files structured in a
different way and store them at different locations, the following subsections
will provide an overview of where the files are located and which tools can
be used to analyze them in Linux and Windows. Moreover, particular attention
will be given to the information relative to USB devices, since it is the main
focus of this study.

\subsection{Windows}\label{sec:litWindows}
In the newest iterations of Microsoft Windows, all the log files are stored by
default in the folder \texttt{\%SystemRoot\%\\system32\\winevt\\Logs}. These
files can be opened with a tool called \emph{Event Viewer} in order to examine
their content. This tool, however, allows only for simple automatic analysis, as
a general keyword search, but it does not automatically extract the most
relevant information stored.

From a forensics investigator perspective, the use of Event Viewer implies a
tedious and error prone manual analysis that is surely not desirable. For this
reason, other tools have been developed that allow the extraction and analysis
of log files from system images in an automatic fashion, so that the
investigator just needs to observe the relevant data and connect them to the
evidence already collected.

The two positive consequences of the usage of these tools are that time is saved
which can be dedicated to other activities and, most importantly, that the
analysis of the evidence can be considered forensically sound. Indeed, in order
to avoid damaging or contaminating the evidence collected, the system analyzed
should always be imaged and the image should be hashed in order to allow for a
subsequent verification of the investigative process. These tools allow for the
analysis of these images directly and guarantee the absence of contamination
during the process.~\cite{murphey2007automated}

The open source tool that is most commonly used for this purpose is called
\emph{Autopsy}, developed by \emph{SleuthKit}; between other things, it also
analyzes the system logs and extracts a list of all the USB devices that were
attached to the machine according to the information in the files. This allows
an investigator retrieve important information, as the serial number of the
device or the time of insertion and deletion. This information is important, for
example, to track USB sticks across different machines.~\cite{deb2015usb}

\subsection{Linux}
Regarding the Linux operating system, the kernel has an internal log on which
every module, and thus also the USB drivers, can write on. This log, called
internal ring buffer, is not directly stored in persistent memory; this task,
indeed is demanded to other services that also collect logs from other sources,
as the \emph{X11} display server, and construct a general system log.

Unfortunately, different distributions use different log-handling daemons and
even different versions of the same daemon which may store the log files in
different locations. The most common ones are the \emph{syslog} daemon, which
stores the log in the \emph{/var/log/syslog} file in plaintext, and the
\emph{journald} daemon, part of the systemd project, which stores the files in
the \emph{/var/log/journal/} directory in binary format. There are also other
interesting combinations that include collecting the data using journald and
then store them in a syslog compatible format, as Ubuntu and derivates
do.~\cite{poettering2012journal}

All the USB related information flows from the kernel ring buffer to the log
daemon and gets stored in one of these locations. Given the variety of locations
and daemons that are currently being used, to the best of our knowledge, there
is no automated tool that is capable of extracting automatically, from a disk
image, information about USB devices; currently, the only option is manual
analysis.

\section{The research preparation}
\subsection{General plan}
For this research we will create a few images of different Linux distributions, which ones will be discussed in section \ref{sec:usedLinux} and how we create them is discussed in section \ref{sec:createImage}. We will then analyze the created images' log files by a python program we wrote, which is explained in section \ref{sec:python}

\subsection{Used Linux distribution systems}\label{sec:usedLinux}
We are using the following Linux distributions. 
\begin{itemize}
\item Fedora 25, kernel 4.8.6
\item OpenSUSE 
\item Ubuntu 
\item Debian 
\end{itemize}
For each Linux we used the last stable version, we start installing the distributions on May 29 2017.
With these distributions mentioned above, we cover most of the popular Linux distro's \cite{LinuxDistro}, as we can see that some other Linux distro's are part of these. Distributions like Elementary, Linux Mint and  Zorin are all Ubuntu based and thus covered in our research.
% other website with top10 distro https://brashear.me/blog/2015/08/24/results-of-the-2015-slash-r-slash-linux-distribution-survey/

\subsection{Creating the images}\label{sec:createImage}
The following steps were taken to prepare for the research:
\begin{itemize}
\item Clear the computer
\item Install the Linux Distribution (default installation, version as mentioned in Section \ref{sec:usedLinux})
\subitem Make sure all drives work correctly (Note: this influence the log file)
\item Then we repeat 5 times, the following sequence
\subitem Start up
\subitem Plug in mouse
\subitem Plug in USB 2.0
\subitem Plug in USB 3.0
\item Create image of the machine
\end{itemize}

\subsection{The program in Python}\label{sec:python}


\section{Found results}


\section{Conclusion}



Analyzing the log fies of a system could besides tracking usb usages, be very useful for other digital forensic applications, like finding unexpected behavior and track the use of the computer.


\bibliographystyle{abbrv}
\bibliography{sample}

\end{document}
